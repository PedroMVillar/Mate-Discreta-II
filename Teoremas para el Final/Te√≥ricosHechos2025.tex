\documentclass[a4paper]{article}

\usepackage{xcolor}
\usepackage[
  nomarginpar,
  top=1in,
  bottom=1in,
  left=1in,
  right=1in
]{geometry} 
\usepackage{amsmath, amssymb}  

% Espaciado entre párrafos y sin sangría
\setlength{\parindent}{0em}
\setlength{\parskip}{1em}

% Contadores de problema y subproblema
\newcounter{problem}
\newcounter{subproblem}[problem]

% Comando \problem (con o sin *)
\makeatletter
\newcommand{\problem}{%
  \@ifstar{\Problem}{\stepcounter{problem}\Problem{\theproblem}}}

\newcommand{\Problem}[1]{%
  \setcounter{problem}{#1}%
  \par
  {\mbox{\textbf{\Large Pregunta #1}}%
  \leaders\hbox{\colorbox{black!20}{\phantom{\Large I}}}\hfill}%
  \hspace{-2em}\colorbox{black!20}{\phantom{\Large I}}%
  \par
}

% Comando \subproblem (con o sin *)
\newcommand{\subproblem}{%
  \@ifstar{\Subproblem}{\stepcounter{subproblem}\Subproblem{\thesubproblem}}}

\newcommand{\Subproblem}[1]{%
  \setcounter{subproblem}{#1}%
  \par\vspace{0.5em}
  {\colorbox{black!10}{\makebox[\dimexpr\linewidth-2\fboxsep]{%
    \textbf{\large Solución}%
  }}}\par
}
\makeatother

\begin{document}

\problem
¿Cuál es la complejidad del algoritmo de Edmonds-Karp? Probarlo. (\textsc{Nota: en la prueba se definen unas distancias, y se prueba que esas distancias no disminuyen en pasos sucesivos de EK. Ud. puede usar esto sin necesidad de probarlo})

\subproblem
Completar prueba

\problem   

Probar que si, dados vértices \(x, z\) y flujo \(f\), definimos la distancia relativa a \(f\) como la longitud del menor \(f\)-camino aumentante entre \(x\) y \(z\) (si existe), o infinito si no existe, o 0 si \(x = z\), denotándola como \(d_f(x, z)\), y definimos \(d_k(x) = d_{f_k}(s, x)\), donde \(f_k\) es el \(k\)-ésimo flujo en una corrida de Edmonds-Karp, entonces \(d_k(x) \leq d_{k+1}(x)\).

\subproblem
Completar prueba

\problem
Probar que si, dados vértices \(x, z\) y flujo \(f\), definimos la distancia relativa a \(f\) como la longitud del menor \(f\)-camino aumentante entre \(x\) y \(z\) (si existe), o infinito si no existe, o 0 si \(x = z\), denotándola como \(d_f(x, z)\), y definimos \(b_k(x) = d_{f_k}(x, t)\), donde \(f_k\) es el \(k\)-ésimo flujo en una corrida de Edmonds-Karp, entonces \(b_k(x) \leq b_{k+1}(x)\).  
\textit{(Este teorema solo se tomará a partir de diciembre 2025).}

\subproblem
Completar prueba

\problem
¿Cuál es la complejidad del algoritmo de Dinic? Probarla en ambas versiones: Dinitz original y Dinic-Even.  
\textit{(No hace falta probar que la distancia en redes auxiliares sucesivos aumenta).}

\subproblem
Completar prueba

\problem
¿Cuál es la complejidad del algoritmo de Wave? Probarla.  
\textit{(No hace falta probar que la distancia en redes auxiliares sucesivos aumenta).}

\subproblem
Completar prueba

\problem
Probar que la distancia en redes auxiliares sucesivos aumenta.  
\textit{(Este teorema solo se tomará a partir de diciembre 2025).}

\subproblem
Completar prueba

\problem
Probar que si \(f\) es un flujo maximal, entonces existe un corte \(S\) tal que \(v(f) = \text{cap}(S)\).  
\textit{(Puede usar sin necesidad de probarlo que si \(f\) es flujo y \(S\) es corte, entonces \(v(f) = f(S, \bar{S}) - f(\bar{S}, S)\)).}

\subproblem
Completar prueba

\problem
Probar que si \(G\) es conexo y no regular, entonces \(\chi(G) \leq \Delta(G)\).

\subproblem
Completar prueba

\problem
Probar que 2-COLOR es polinomial.

\subproblem
Completar prueba

\problem
Enunciar y probar el Teorema de Hall.

\subproblem
Completar prueba

\problem
Enunciar y probar el teorema del matrimonio de Kőnig.

\subproblem
Completar prueba

\problem
Probar que si \(G\) es bipartito entonces \(\chi'(G) = \Delta(G)\).  
\textit{(Este teorema solo se tomará a partir de diciembre 2025).}

\subproblem
Completar prueba

\problem
Demostrar las complejidades \(O(n^4)\) y \(O(n^3)\) del algoritmo Húngaro.  
\textit{(Solo a partir de diciembre 2025).}

\subproblem
Completar prueba

\problem
Enunciar el teorema de la cota de Hamming y probarlo.

\subproblem
Completar prueba

\problem
Probar que si \(H\) es matriz de chequeo de \(C\), entonces:  
\[
\delta(C) = \min \{ j \mid \exists \text{ un conjunto de } j \text{ columnas LD de } H \}
\]  
(LD significa “linealmente dependiente”).

\subproblem
Completar prueba


\problem
Sea \(C\) un código cíclico de dimensión \(k\) y longitud \(n\), y sea \(g(x)\) su polinomio generador. Probar que:  

\begin{itemize}
    \item[i)] \(C\) está formado por los múltiplos de \(g(x)\) de grado menor que \(n\).
    \item[ii)] $ C = \{ v(x) \cdot g(x) : v(x) \text{ es un polinomio cualquiera} \} $
    \item[iii)] \(gr(g(x)) = n - k\)
    \item[iv)] \(g(x)\) divide a \(1 + x^n\)
\end{itemize}

\subproblem
Completar prueba

\problem
Probar que 3SAT es NP-completo.

\subproblem
Completar prueba

\problem
Probar que 3-COLOR es NP-completo.

\subproblem
Completar prueba

\problem
Probar que Matrimonio3D (matrimonio trisexual) es NP-completo.

\subproblem
Completar prueba


\end{document}

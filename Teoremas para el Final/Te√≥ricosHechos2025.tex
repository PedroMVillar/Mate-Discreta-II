\documentclass[a4paper]{article}

\usepackage{xcolor}
\usepackage[
  nomarginpar,
  top=1in,
  bottom=1in,
  left=1in,
  right=1in
]{geometry} 
\usepackage{amsmath, amssymb}  
\usepackage{algorithm}
\usepackage{algorithmic}

% Espaciado entre párrafos y sin sangría
\setlength{\parindent}{0em}
\setlength{\parskip}{1em}

% Contadores de problema y subproblema
\newcounter{problem}
\newcounter{subproblem}[problem]

% Comando \problem (con o sin *)
\makeatletter
\newcommand{\problem}{%
  \@ifstar{\Problem}{\stepcounter{problem}\Problem{\theproblem}}}

\newcommand{\Problem}[1]{%
  \setcounter{problem}{#1}%
  \par
  {\mbox{\textbf{\Large Pregunta #1}}%
  \leaders\hbox{\colorbox{black!20}{\phantom{\Large I}}}\hfill}%
  \hspace{-2em}\colorbox{black!20}{\phantom{\Large I}}%
  \par
}

% Comando \subproblem (con o sin *)
\newcommand{\subproblem}{%
  \@ifstar{\Subproblem}{\stepcounter{subproblem}\Subproblem{\thesubproblem}}}

\newcommand{\Subproblem}[1]{%
  \setcounter{subproblem}{#1}%
  \par\vspace{0.5em}
  {\colorbox{black!10}{\makebox[\dimexpr\linewidth-2\fboxsep]{%
    \textbf{\large Solución}%
  }}}\par
}
\makeatother

\begin{document}

\problem
¿Cuál es la complejidad del algoritmo de Edmonds-Karp? Probarlo. (\textsc{Nota: en la prueba se definen unas distancias, y se prueba que esas distancias no disminuyen en pasos sucesivos de EK. Ud. puede usar esto sin necesidad de probarlo})

\subproblem
Completar prueba

\problem   

Probar que si, dados vértices \(x, z\) y flujo \(f\), definimos la distancia relativa a \(f\) como la longitud del menor \(f\)-camino aumentante entre \(x\) y \(z\) (si existe), o infinito si no existe, o 0 si \(x = z\), denotándola como \(d_f(x, z)\), y definimos \(d_k(x) = d_{f_k}(s, x)\), donde \(f_k\) es el \(k\)-ésimo flujo en una corrida de Edmonds-Karp, entonces \(d_k(x) \leq d_{k+1}(x)\).

\subproblem
Completar prueba

\problem
Probar que si, dados vértices \(x, z\) y flujo \(f\), definimos la distancia relativa a \(f\) como la longitud del menor \(f\)-camino aumentante entre \(x\) y \(z\) (si existe), o infinito si no existe, o 0 si \(x = z\), denotándola como \(d_f(x, z)\), y definimos \(b_k(x) = d_{f_k}(x, t)\), donde \(f_k\) es el \(k\)-ésimo flujo en una corrida de Edmonds-Karp, entonces \(b_k(x) \leq b_{k+1}(x)\).  
\textit{(Este teorema solo se tomará a partir de diciembre 2025).}

\subproblem
Completar prueba

\problem
¿Cuál es la complejidad del algoritmo de Dinic? Probarla en ambas versiones: Dinitz original y Dinic-Even.  
\textit{(No hace falta probar que la distancia en redes auxiliares sucesivos aumenta).}

\subproblem
Completar prueba

\problem
¿Cuál es la complejidad del algoritmo de Wave? Probarla.  
\textit{(No hace falta probar que la distancia en redes auxiliares sucesivos aumenta).}

\subproblem
Completar prueba

\problem
Probar que la distancia en redes auxiliares sucesivos aumenta.  
\textit{(Este teorema solo se tomará a partir de diciembre 2025).}

\subproblem
Completar prueba

\problem
Probar que si \(f\) es un flujo maximal, entonces existe un corte \(S\) tal que \(v(f) = \text{cap}(S)\).  
\textit{(Puede usar sin necesidad de probarlo que si \(f\) es flujo y \(S\) es corte, entonces \(v(f) = f(S, \bar{S}) - f(\bar{S}, S)\)).}

\subproblem
Definamos
$$
S=\{s\} \cup\{x \in V: \text { exista un } f \text {-camino aumentante desde } s \text { a } x\}
$$

\begin{itemize}
    \item Como $f$ es maximal entonces no existen $f$-caminos aumentantes desde $s$ a $t$ pues si existiese un tal $f$-camino aumentante, podriamos mandar un $\varepsilon>0$ a través de el, obteniendo un flujo $f^{*}$ tal que $v\left(f^{*}\right)=v(f)+\varepsilon>v(f)$ lo cual contradice que $f$ sea maximal. Por lo tanto, como no existen $f$-caminos aumentantes desde $s$ a $t$ entonces $t \notin S$.
    \item Como $s \in S$ y  $t \notin S$, entonces $S$ \textbf{es un corte}.
\end{itemize}

Observemos que en el resto de la prueba no usaremos que $f$ es maximal, solo que es flujo y que $S$ es corte. Es decir, la prueba valdría para cualquier $f$ tal que el $S$ definido sea un corte. Esto será importante luego.

Como $f$ es flujo y $S$ es corte, entonces $v(f)=f(S, \bar{S})-f(\bar{S}, S)$. Calculemos $f(S, \bar{S})$ y $f(\bar{S}, S)$.

\begin{enumerate}
    \item Cálculo de $f(S, \bar{S})$:
    $$
    f(S, \bar{S})=\sum_{x, y} f(\overrightarrow{x y})[x \in S][y \notin S][x y \in E]
    $$
    Consideremos un par $x, y$ de los que aparecen en esa suma. Como $x \in S$, entonces existe un $f$-camino aumentante entre $s$ y $x$, digamos $s=$ $x_{0}, x_{1}, \ldots, x_{r}=x$. Pero como $y \notin S$, entonces no existe ningún $f$-camino aumentante entre $s$ e $y$. En particular, el camino
    $$
    s=x_{0}, x_{1}, \ldots, x_{r}=x, y
    $$
    \textbf{no es un $f$-camino aumentante}. Pero $\overrightarrow{x y} \in E$, asi \underline{que debería poder serlo}.
    
    ¿Por qué $s=x_{0}, x_{1}, \ldots, x_{r}=x, y$ no es un $f$-camino aumentante a pesar de que $s=x_{0}, x_{1}, \ldots, x_{r}=x$ si lo es y $\overrightarrow{x y}$ existe? La única razón por la cual no es un $f$-camino aumentante es porque no podemos usar el lado $\overrightarrow{x y}$ por estar saturado, es decir:
    $$
    f(\overrightarrow{x y})=c(\overrightarrow{x y})
    $$
    Esto es cierto para cualesquiera $x, y$ que aparezcan en esa suma. Entonces:
    $$
    \begin{aligned}
    f(S, \bar{S}) & =\sum_{x, y} f(\overrightarrow{x y})[x \in S][y \notin S][x y \in E] \\
    & =\sum_{x, y} c(\overrightarrow{x y})[x \in S][y \notin S][x y \in E] \\
    & =c(S, \bar{S})=\operatorname{cap}(S)
    \end{aligned}
    $$
    \item Cálculo de $f(\bar{S}, S)$:
    $$
    f(\bar{S}, S)=\sum_{x, y} f(\overrightarrow{x y})[x \notin S][y \in S][x y \in E]
    $$
    Consideremos un par $x, y$ de los que aparecen en esa suma. Como $y \in S$, entonces existe un $f$-camino aumentante entre $s$ e $y$, digamos $s=x_{0}, x_{1}, \ldots, x_{r}=y$. Pero como $x \notin S$, entonces no existe un $f$-camino aumentante entre $s$ y $x$. En particular
    $$
    s=x_{0}, x_{1}, \ldots, x_{r}=y, x
    $$
    NO ES un $f$-camino aumentante. Pero $\overrightarrow{x y} \in E$, asi que PODRIA serlo, usando $y, x$ como lado backward.

    ¿Porqué $s=x_{0}, x_{1}, \ldots, x_{r}=y, x$ no es un $f$-camino aumentante a pesar de que $s=x_{0}, x_{1}, \ldots, x_{r}=y$ si lo es y $\overrightarrow{x y}$ existe? La única razón es que no podemos usarlo como lado backward por estar vacio, es decir, que $f(\overrightarrow{x y})=0$.

    Esto es cierto para cualesquiera $x, y$ que aparezcan en esa suma. Entonces:

    $$
    \begin{aligned}
    f(\bar{S}, S) & =\sum_{x, y} f(\overrightarrow{x y})[x \notin S][y \in S][x y \in E] \\
    & =\sum_{x, y} 0[x \notin S][y \in S][x y \in E] \\
    & =0
    \end{aligned}
    $$
\end{enumerate}
Entonces probamos que para este $S$:
\begin{itemize}
    \item $f(S, \bar{S})=\operatorname{cap}(S)$
    \item $f(\bar{S}, S)=0$
\end{itemize}
Por lo tanto
$$
\begin{aligned}
v(f) & =f(S, \bar{S})-f(\bar{S}, S) \\
& = \operatorname{cap}(S)-0=\operatorname{cap}(S)
\end{aligned}
$$

\problem
Probar que si \(G\) es conexo y no regular, entonces \(\chi(G) \leq \Delta(G)\).

\subproblem
Sea $G$ un grafo conexo con grado máximo $\Delta(G)$. Si $G$ es un ciclo impar o un grafo completo, es conocido que $\chi(G) = \Delta(G) + 1$. Supongamos que $G$ no es ninguno de estos casos y probemos que $\chi(G) \leq \Delta(G)$.

Primero vamos a plantear la estrategia de la demostración:

\begin{enumerate}
    \item Elegimos un vértice $x$ de grado mínimo $\delta$.
    \item Luego, con ese vértice ejecutamos BFS, para geneerar un orden de los vértices.
    \item Tomamos el orden inverso al obtenido y coloreamos usando Greedy.
    \item Mostramos que en cada paso hay a lo sumo $\Delta - 1$ colores ocupados en los vecinos anteriores.
    \item Con $\Delta$ colores disponibles, siempre hay un color libre.
\end{enumerate}

Elegimos un vértice arbitrario $x$ (tal que $d(x) = \delta(x)$) y ejecutamos un \textit{BFS} desde $x$. Esto nos da un árbol generador de $G$ y un orden en el que se descubren los vértices. Denotemos este orden como:
\[
y_1, y_2, \dots, y_n, \quad \text{donde } y_1 = x.
\]
Invertimos este orden y llamamos a la secuencia resultante:
\[
x_1, x_2, \dots, x_n, \quad \text{donde } x_n = x.
\]

En este nuevo orden, salvo por $x_n = x$, cada vértice $x_i$ tiene al menos un vecino posterior en la secuencia. Esto es porque en el BFS cada vértice (salvo la raíz) es descubierto por algún vecino anterior en el BFS, lo que se traduce en un vecino posterior en el orden invertido.

Utilizamos un coloreo greedy en el orden $x_1, x_2, \dots, x_n$:
\begin{itemize}
    \item Asignamos el color 1 a $x_1$.
    \item Para cada $x_i$ con $i > 1$, elegimos el menor color disponible que no haya sido asignado a un vecino anterior en el orden.
\end{itemize}

Observamos que, al momento de colorear $x_i$, la cantidad de colores que ya están usados en sus vecinos anteriores es a lo sumo $\Delta(G) - 1$. Esto se debe a que $x_i$ siempre tiene al menos un vecino posterior en el orden, por lo que no todos sus $\Delta(G)$ vecinos pueden haber sido coloreados antes que él.

Dado que hay $\Delta(G)$ colores disponibles, siempre hay un color libre para cada vértice.

El coloreo greedy nunca usa más de $\Delta(G)$ colores, por lo que $\chi(G) \leq \Delta(G)$, como queríamos demostrar.

\problem
Probar que 2-COLOR es polinomial.

\subproblem
Para probar esto, vamos a dar un algoritmo, que cumpla lo siguiente:
\begin{enumerate}
    \item Resuelve el problema de $2$-color.
    \item Corre en tiempo polinomial.
\end{enumerate}

\begin{center}
    La idea es dar primero el algoritmo, luego probar que es polinomial y por último en la \texttt{correctitud}, probar que el algoritmo da respuestas correctas, es decir, funciona.
\end{center}

Antes de comenzar con la prueba, vamos a dar una observación que nos será útil mas adelante:
\begin{itemize}
    \item \textbf{$\chi(G) \leq 2 \iff \chi(C) \leq 2 \quad \forall c.c \in C$ del grafo $G$, es decir, si cada componente conexa del grafo tiene número cromático menor o igual a $2$, entonces el grafo en sí también lo tiene.}
\end{itemize}
\textbf{[-]} Para resolver el problema de $2$-color, vamos a usar el algoritmo de BFS, que nos permite colorear un grafo de manera eficiente. La idea es la siguiente:
\begin{itemize}
    \item Tomo un vértice $x$ cualquiera en un grafo conexo $G$.
    \item Construyo un árbol generador $T$ de $G$ usando BFS con raíz en $x$.
    \item Ejecutar BFS desde $x$ y determinar el nivel de cada vértice $z$.
    \item Luego asigna los colores según el nivel de los vértices, si el nivel es impar, asigno color $1$, si es par, asigno color $2$.
    \item Por último verifico que no haya aristas que conecten vértices del mismo color, si las hay, entonces el grafo no es $2$-coloreable.
\end{itemize}
\begin{algorithm}
    \caption{Verificar si un grafo es bipartito (2-color)}
    \begin{algorithmic}[1]
        \STATE \textbf{Entrada:} Grafo $G = (V, E)$ conexo
        \STATE \textbf{Salida:} "Sí" si $G$ es bipartito, "No" en caso contrario
        \STATE Seleccionar un vértice inicial $x \in V(G)$
        \STATE Construir un árbol generador $T$ de $G$ usando BFS con raíz en $x$
        \STATE Ejecutar BFS desde $x$ y determinar el nivel de cada vértice $z$
        \FOR{cada vértice $z \in V(G)$}
            \IF{nivel de $T(z)$ es impar}
                \STATE Asignar $c(z) \gets 1$
            \ELSE
                \STATE Asignar $c(z) \gets 2$
            \ENDIF
        \ENDFOR
        \FOR{cada arista $(u, v) \in E(G)$}
            \IF{$c(u) = c(v)$}
                \STATE \textbf{Retornar} "No" \COMMENT{El grafo no es bipartito}
            \ENDIF
        \ENDFOR
        \STATE \textbf{Retornar} "Sí" \COMMENT{El grafo es bipartito}
    \end{algorithmic}
\end{algorithm}
\textbf{[2.]} Ahora tenemos que probar la complejidad, como ya sabemos el colorear es gratis, el peso de complejidad recae en el BFS y en el chequeo de las aristas, ambos corren en tiempo $O(m)$ donde $m$ es la cantidad de aristas, por lo tanto, el algoritmo corre en tiempo polinomial.

Ahora bien, si el algoritmo anterior devuelve \textbf{Sí}, es porque el coloreo es propio, y por lo tanto $\chi(G) \leq 2$. En caso contrario, $\chi(G) \geq 3$.

Para probar esto, vamos a ver que el grafo $G$ \textbf{tiene un ciclo impar}. Para comenzar, voy a plantear la estrategia de la demostración:

\textbf{[1]} Por hipótesis, tenemos que el coloreo del \textit{ciclo for} de nuestro algoritmo \textbf{no es propio}. Esto quiere decir que existe una arista $zv \in V(G)$ tal que $c(z) = c(v)$, en este dato vamos a basar la demostración.

\begin{enumerate}
    \item Considerar el spanning tree $T$ de $G$ generado a partir de un vértice $x$.
    \item Analizar los caminos únicos de $x$ a $z$ y $v$.
    \item Usar la arista $zv$ para construir un ciclo impar.
    \item Probar que el ciclo es impar.
\end{enumerate}

Sea $T$ el spanning tree con raíz $x$. Como $T$ es un árbol, hay un único camino desde $x$ a cualquier otro vértice.
\begin{itemize}
    \item \textbf{Camino de $x$ a $z$ en $T$}: $xz_1z_2 \ldots z_j $ y el nivel de $z$ es $j$.
    \item \textbf{Camino de $x$ a $v$}: $xv_1v_2 \ldots v_i$ y el nivel de $v$ es $i$.
\end{itemize}

Como el coloreo no es propio, entonces $c(z) = c(v)$, lo que implica que ambos son pares o ambos son impares, por lo tanto \textbf{la suma de los niveles es par} (i+j es par). 

Dado que ambos caminos empiezan desde el mismo punto $x$ y terminan en distintos puntos $z$ y $v$, en algún punto se separan en $T$, sea $w$ el último vértice común de ambos caminos, de forma tal que:

\begin{equation*}
    z_0=v_0, \, z_1=v_1, \ldots, z_p=v_p = w \, 
\end{equation*}

Teniendo en cuenta que $xv$ es un lado de $G$ y que $z_k = z$ y $v_j = v$, entonces en $G$ tenemos el ciclo:
\begin{equation*}
    C = w\underbrace{z_{p+1}z_{p+2} \ldots z_{k-1}z}_{k-p \, \, \, \text{vértices}}\underbrace{vv_{j-1}\ldots v_{p+1}}_{j-p \, \, \, \text{vértices}} w
\end{equation*}
en total $C$ tiene $1+k-p+j-p$ vértices, es decir, $k+j-2p+1$ vértices. Como $i+j$ es par, entonces $k+j-2p+1$ es impar, por lo tanto $C$ es un ciclo impar.

\problem
Enunciar y probar el Teorema de Hall.

\subproblem
Completar prueba

\problem
Enunciar y probar el teorema del matrimonio de Kőnig.

\subproblem
Completar prueba

\problem
Probar que si \(G\) es bipartito entonces \(\chi'(G) = \Delta(G)\).  
\textit{(Este teorema solo se tomará a partir de diciembre 2025).}

\subproblem
Completar prueba

\problem
Demostrar las complejidades \(O(n^4)\) y \(O(n^3)\) del algoritmo Húngaro.  
\textit{(Solo a partir de diciembre 2025).}

\subproblem
Completar prueba

\problem
Enunciar el teorema de la cota de Hamming y probarlo.

\subproblem
Completar prueba

\problem
Probar que si \(H\) es matriz de chequeo de \(C\), entonces:  
\[
\delta(C) = \min \{ j \mid \exists \text{ un conjunto de } j \text{ columnas LD de } H \}
\]  
(LD significa “linealmente dependiente”).

\subproblem
Completar prueba


\problem
Sea \(C\) un código cíclico de dimensión \(k\) y longitud \(n\), y sea \(g(x)\) su polinomio generador. Probar que:  

\begin{itemize}
    \item[i)] \(C\) está formado por los múltiplos de \(g(x)\) de grado menor que \(n\).
    \item[ii)] $ C = \{ v(x) \cdot g(x) : v(x) \text{ es un polinomio cualquiera} \} $
    \item[iii)] \(gr(g(x)) = n - k\)
    \item[iv)] \(g(x)\) divide a \(1 + x^n\)
\end{itemize}

\subproblem
Completar prueba

\problem
Probar que 3SAT es NP-completo.

\subproblem
Completar prueba

\problem
Probar que 3-COLOR es NP-completo.

\subproblem
Completar prueba

\problem
Probar que Matrimonio3D (matrimonio trisexual) es NP-completo.

\subproblem
Completar prueba


\end{document}
